\documentclass[a4paper,12pt]{article}

\usepackage[utf8]{inputenc}
\usepackage[english]{babel}
\usepackage{natbib}
\usepackage[T1]{fontenc}
\usepackage{setspace}


\title{Introduction}
\author{Joshua Bridge \\14032908 \\joshua.m.bridge@stu.mmu.ac.uk}


\begin{document}

\maketitle

\tableofcontents

\doublespacing

 % TODO Complete summary of introduction

\section{Digital image processing}
  Image processing deals with the analysis and manipulation of image data. An example of image processing is the use of digital signal processing. This involves converting analog sensory data from a digital camera sensor into a computer-interpretable format with minimal data loss from external sources such as noise and distortion.\\

  \subsection{Bitmaps}
    Before you are able to analyse an image, you must first represent the data in a way that it can be interpreted by a computer, and a human. One basic form of doing this is via a bitmap image. A bitmap - as its name implies - is a simple spacial mapping of values (bits) along a horizontal axis (x) and vertical axis (y). Using a greyscale image as an example, a bitmap representation of this would contain a number of values (or ‘pixels’), the number of which is equal to the product of the sizes of the x and y axis. Therefore an image of size 200 x 200 would contain 40,000 pixels. Each of these pixels contains an integer value representing brightness, typically ranging from 0 - 255 (the total value range of an 8-bit integer), ‘0’ being completely black, ‘255’ being completely white.

    A colour image follows a very similar format, except now each pixel contains three brightness values instead of one. Each of these values map to the brightness of the colours (or ‘channels’) red, green and blue - in that order. Therefore a pixel with values (0,255,0) would be entirely green and a pixel with values (0,0,255) would be entirely blue. It should be noted that when these colours are displayed on a computer screen their colour values are additive (i.e. they can mix together to form a different, brighter colour). A pixel with values (0,255,255) would therefore represent cyan, and finally a pixel with values (255,255,255) would represent white.

  \subsection{Image processing tasks}
    There are several methods of improving the results of image analysis, one of which is to run an image processing algorithm against it. Genereally this is to make the image clearer by sharpening it or removing noise - these are often referred to as low-level processing methods \citep{sonka2014image}. While these methods are often applied to make analysis by a computer a lot easier, they also can be used to increase the ‘clarity’ or the percieved ‘beauty’ of an image when viewed by a human.

    \subsubsection{Sharpening}
      For an image to be captured, it must first enter through some kind of lens which refracts the incoming light into a ‘focal point’ onto which some kind of light-sensitive surface is placed such as a digital image sensor. In order for an object to be perfectly ‘in focus’ it must be at the optimal distance from the lens - an area known as the ‘focal plane’. If the subject of a photograph is not near enough to the focal plane (either in front of it or behind it) then the subject will appear to blur.

\section{Computer vision}
  Computer vision involves modelling the human vision system in such a way that a computer can interpret abstract visual data.

\bibliographystyle{agsm}
\bibliography{introduction}

\end{document}
