\documentclass[a4paper,12pt]{article}

\usepackage[utf8]{inputenc}
\usepackage[english]{babel}
\usepackage{natbib}
\usepackage[T1]{fontenc}
\usepackage{setspace}


\title{Literature Review}
\author{Joshua Bridge \\14032908 \\joshua.m.bridge@stu.mmu.ac.uk}


\begin{document}

\maketitle

\tableofcontents

\doublespacing

\section{Introduction}
  In order to understand the background of the technology involved in this project, it is necessary to complete a significant review into the current and past literature. This will help form a basis of knowledge from which the future development and analysis of the proposed system will utilise and build upon. This research will be vital in order to make use of the most optimal technology for any given problem.

  The following review will be an investigation into the current knowledge of the three major components of the proposed system, which are as follows:

  \begin{itemize}
    \item The recommendation system.
    \item The image-filter generation system.
    \item The web-based user interface.
  \end{itemize}

\section{Recommendation systems}
  As defined in \cite{ricci2011introduction}, recommending content is a problem which involves attempting to predict what a user may desire at any given time when using a system. \cite{jannach2010recommender} describe four different methods for giving recommendations: Collaborative, Content-based, Knowledge-based and Hybrid.

  \subsection{Collaborative recommendation}

  \subsection{Content-based recommendation}

  \subsection{Knowledge-based recommendation}
    The information needed to carry out such a prediction can include the users history of interaction with the system, from which preference can be extrapolated (a.k.a. ‘metadata’). If there is not enough metadata about the user from which to extrapolate preference with reasonable certainty, then preference can be inferred from other users of the system - especially users with a similar predicted preference.

  \subsection{Hybrid recommendation}

  \subsection{Machine learning recommendation}

  \subsection{Image classification in recommendations}

\section{Image-filter generation}

\section{Web-based user interface}

\section{Conclusion}

\bibliographystyle{agsm}
\bibliography{literature-review}


\end{document}
